\documentclass{article}

\usepackage{times}

%\fontencoding{T1}
%\fontfamily{times}
%\fontsize{12}{15}
%\selectfont

\setcounter{secnumdepth}{0}
\setcounter{tocdepth}{2}

\newcommand{\parameter}[2]{\noindent \texttt{#1}: #2}
\newcommand{\function}[4]{
  \subsection{\uppercase{#1}}

  #2

  \subsubsection{Parameters}
  #3

  \subsubsection{Description}
  #4
}

\title{Fusion-IO Reference}
\author{Nathaniel Ferraro}

\begin{document}

\maketitle

\tableofcontents

\section{Introduction}

The Fusion-IO (FIO) library is an interface for extracting data from the
output of various fluid plasma codes in a code-independent way.  The
library is built to read data directly from the native output of these
codes, and therefore no common data format is defined or required.
This strategy avoids the loss of information incurred by converting
the native code data into a common format (\textit{e.g.} interpolation
error), and preserves the properties of the computed solution
(\textit{e.g.} divergence-free fields) exactly.

FIO provides access to data through several structures:
\begin{description}
\item[Source]: provides access to a data file;
\item[Field]: allows evaluation of a physical field in real space;
\item[Series]: allows evaluation of a time series
\item[Parameter]: allows evaluation of specific properties of a field,
  source, or series.
\end{description}
A \textit{source} structure should always be created first to provide
access to a data source, then \textit{fields}, \textit{series}, and
\textit{parameters} can be read from that data source and evaluated.

All input and output data representing physical quantities is in SI units.

\section{Field Functions}

\function{fio\_close\_field}{
\noindent
int ierr = fio\_close\_field(const int ifield);\\

\noindent
call fio\_close\_field\_f(ifield, ierr)\\
integer, intent(in) :: ifield\\
integer, intent(out) :: ierr
}
{
\parameter{ifield}{Field handle returned by \texttt{fio\_get\_field}.}
}
{
Deallocates data associated with field \texttt{ifield}
}


\function{fio\_eval\_field}{
\noindent
int ierr = fio\_eval\_field(const int ifield, const double* x, double* f);\\

\noindent
call fio\_eval\_field\_f(ifield, x, f, ierr)
integer, intent(in) :: ifield\\
real, intent(in), dimension(*) :: x\\
real, intent(out), dimension(*) :: f\\
integer, intent(out) :: ierr
}
{
\parameter{ifield}{Field handle returned by \texttt{fio\_get\_field}.}\\
\parameter{x}{Three-element array containing $(R, \varphi, Z)$ coordinates at which to evaluate field.}\\
\parameter{f}{Array containing field data.  For scalar fields, this is an array of length 1; for vector fields, this is an array of length 3 containing the $(R, \varphi, Z)$ components of the vector.}
}
{
Evaluates a field at a given $(R, \varphi, Z)$ coordinate.
}

\function{fio\_eval\_field\_deriv}{
\noindent
int ierr = fio\_eval\_field\_deriv(const int ifield, const double* x,
double* df);\\

\noindent
call fio\_eval\_field\_deriv\_f(ifield, x, df, ierr)\\
integer, intent(in) :: ifield\\
real, intent(in), dimension(*) :: x\\
real, intent(out), dimension(*) :: df\\
integer, intent(out) :: ierr
}
{
\parameter{ifield}{Field handle returned by \texttt{fio\_get\_field}.}\\
\parameter{x}{Three-element array containing $(R, \varphi, Z)$
  coordinates at which to evaluate field derivatives.}\\
\parameter{df}{Array containing field derivative data data.  For scalar
  fields, this is an array of length 3 containing $(\partial_R f,
  \partial_\varphi f, \partial_Z f)$; for vector fields, this is an
  array of length 9 containing $(\partial_R f_R, \partial_R
  f_\varphi, \partial_R f_Z, 
  \partial_\varphi f_R, \partial_\varphi f_\varphi, \partial_\varphi f_Z,
  \partial_Z f_R, \partial_Z f_\varphi, \partial_Z f_Z)$.}
}
{
  Evaluates the partial derivatives of a field at a given $(R,
  \varphi, Z)$ coordinate.  Note that this does NOT return $\nabla f$.
}




\function{fio\_get\_field}
{
\noindent
int ierr = fio\_get\_field(const int isource, const int ifield\_type, int* ifield);\\

\noindent
call fio\_get\_field\_f(isource, ifield\_type, ifield, ierr)\\
integer, intent(in) :: isource\\
integer, intent(in) :: ifield\_type\\
integer, intent(out) :: ifield\\
integer, intent(out) :: ierr
}
{
\parameter{ifield\_type}{Field to open.  Possibilities include:

\paragraph{Scalar Fields}
\begin{description}
\item[FIO\_SCALAR\_POTENTIAL]: Scalar potential
\item[FIO\_TOTAL\_PRESSURE]: Total pressure
\end{description}

\paragraph{Species-depdentent Scalar Fields}
\begin{description}
\item[FIO\_PRESSURE]: Partial pressure of species
\item[FIO\_DENSITY]: Number density of species
\end{description}

\paragraph{Vector Fields}
\begin{description}
\item[FIO\_ELECTRIC\_FIELD]: Electric field
\item[FIO\_VECTOR\_POTENTIAL]: Electromagnetic vector potential
\item[FIO\_MAGNETIC\_FIELD]: Magnetic field
\item[FIO\_CURRENT\_DENSITY]: Current density
\item[FIO\_FLUID\_VELOCITY]: Barycentric velocity
\end{description}

\paragraph{Species-dependent Vector Fields}
\begin{description}
\item[FIO\_VELOCITY]: Fluid velocity of species
\end{description}
}
\parameter{ifield}{Field handle returned by \texttt{fio\_get\_field}.}
}
{
Provides a handle to evaluate field data.
}


\section{Series Functions}

\function{fio\_close\_series}
{
\noindent
int ierr = fio\_close\_series(const int iseries);\\

\noindent
call fio\_close\_series\_f(iseries, ierr)\\
integer, intent(in) :: iseries\\
integer, intent(out) :: ierr
}
{
  \parameter{iseries}{Field handle returned by \texttt{fio\_get\_series}.}
}
{
  Deallocates data associated with \texttt{iseries}.
}

\function{fio\_eval\_series}
{
  \noindent
  int ierr = fio\_eval\_series(const int iseries, const double t,
  double* s);\\

  \noindent
  call fio\_eval\_series\_f(iseries, t, s, ierr)\\
  integer, intent(in) :: iseries\\
  real, intent(in) :: t\\
  real, intent(out) :: s\\
  integer, intent(out) :: ierr
}
{
  \parameter{iseries}{Handle to series returned by
    \texttt{fio\_get\_series}.}

  \parameter{t}{Time at which to evaluate series.}

  \parameter{s}{Value of series at time \texttt{t}.}
}

\function{fio\_get\_series}
{
\noindent
int ierr = fio\_get\_series(const int isource, const int
iseries\_type, int* iseries);\\

\noindent
call fio\_get\_series\_f(isource, iseries\_type, iseries, ierr)\\
integer, intent(in) :: isource\\
integer, intent(in) :: iseries\_type\\
integer, intent(out) :: iseries\\
integer, intent(out) :: ierr
}
{
\parameter{isource}{The handle to the source file returned by
  \texttt{fio\_open\_source}.}

\parameter{iseries\_type}{The series to open.  Choices include:
\begin{description}
\item[LCFS\_PSI]: Poloidal flux at the last closed flux surface.
\item[MAGAXIS\_PSI]: Poloidal flux at the magnetic axis.
\item[MAGAXIS\_R]: $R$-coordinate of the magnetic axis.
\item[MAGAXIS\_Z]: $Z$-coordinate of the magnetic axis.
\end{description}}

\parameter{iseries}{The handle to the series data.}
}
{
  Returns a handle for accessing data for a time series.
}

\function{fio\_get\_series\_bounds}
{
\noindent
int ierr = fio\_get\_series\_bounds(const int iseries, double* tmin,
double* tmax);\\

\noindent
call fio\_get\_series\_bounds\_f(iseries, tmin, tmax, ierr)\\
integer, intent(in) :: iseries\\
real, intent(out) :: tmin\\
real, intent(out) :: tmax\\
integer, intent(out) :: ierr
}
{
  \parameter{iseries}{Handle to series returned by
    \texttt{fio\_get\_series}.}

  \parameter{tmin}{Earliest available time in time series}

  \parameter{tmax}{Latest available time in time series}
}
{
  Returns the bounds of the domain of a time series.
}


\section{Source Functions}

\function{fio\_close\_source}
{
\noindent
int ierr = fio\_close\_source(const int isource);\\

\noindent
call fio\_close\_source\_f(isource, ierr)\\
integer, intent(in) :: isource\\
integer, intent(out) :: ierr
}
{
  \parameter{isource}{Source handle returned by \texttt{fio\_open\_source}}.
}
{
  Closes file and deallocates resources associated with open source.
}


\function{fio\_open\_source}
{
  \noindent
  int ierr = fio\_open\_source(const int isource\_type, const char* filename, int* isource);\\

  \noindent
  call fio\_open\_source\_f(isource\_type, filename, isource, ierr)\\
  integer, intent(in) :: isource\_type\\
  character(len=*), intent(in) :: filename\\
  integer, intent(out) :: ierr
}
{
  \parameter{isource\_type}{The type of data source to open.  Choices include:
    \begin{description}
    \item[FIO\_GATO\_SOURCE]: (not yet implemented)
    \item[FIO\_GEQDSK\_SOURCE]:
    \item[FIO\_M3DC1\_SOURCE]:
    \item[FIO\_MARS\_SOURCE]: (not yet implemented)
    \item[FIO\_NIMROD\_SOURCE]: (not yet implemented)
    \end{description}
  }

  \parameter{filename}{The name of the data source file.}

  \parameter{isource}{The handle to the data source.}
}
{
  Opens a data source.
}

\section{Not Yet Documented}

int fio\_get\_options(const int);
int fio\_get\_available\_fields(const int, int*, int**);

int fio\_set\_int\_option(const int, const int);
int fio\_set\_str\_option(const int, const char*);
int fio\_set\_real\_option(const int, const double);
int fio\_get\_int\_parameter(const int, const int,  int*);
int fio\_get\_real\_parameter(const int, const int, double*);
int fio\_get\_real\_field\_parameter(const int, const int, double*);

\section{Error Codes}

\begin{description}
  \item[FIO\_SUCCESS (0)]:  Function call returned without errors.
  \item[FIO\_UNSUPPORTED (10001)]: The requested field, series,
    parameter is not included in the source data or is not yet
    implemented in the FIO library.
  \item[FIO\_OUT\_OF\_BOUNDS (10002)]: The requested coordinates are
    outside of the domain of the source data.
  \item[FIO\_FILE\_ERROR (10003)]: The requested source data could not be succesfully read.
  \item[FIO\_BAD\_DIMENSIONS (10004)]: 
  \item[FIO\_BAD\_SPECIES (10005)]: Data for requested species is not
    present in source data.
  \item[FIO\_NO\_DATA (10006)]: Data expected to be found in the source
    data was not present.
\end{description}


\end{document}
